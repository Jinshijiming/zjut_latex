%-------------------------------------------------
% FileName: chapt-3.tex
% Version: 0.1
% Date: 2023-06-25
% Description: 第3章
% Others: 
% History: origin
%------------------------------------------------- 

% 断页
% \clearpage 

\chapter{系统分析(需求分析)}

\section{功能需求分析}
% 引用图片的例子
描述系统的功能性需求,可以通过数据流图或UML的用例图等图表工具来部来定义系统的功能需求,并把需求和设计完全分离开。如图\ref{fig:single}所示。

% xelatex 支持的图片格式
% 矢量图 .pdf .eps 
% 位图 .jpg .png .bmp

% figure环境
% [H] 浮动优先级,当前位置,但尺寸过大的浮动体可能使得分页比较困难

% [htbp!] 浮动方式 请参考一份(不太)简短的 LATEX 2" 介绍,3.9节
% h 当前位置(代码所处的上下文)
% t 顶部
% b 底部
% p 单独成页
% ! 在决定位置时忽视限制
% 排版位置的选取与参数里符号的顺序无关, 
% LATEX 总是以 h-t-b-p 的优先级顺序决定浮动体位置。
% 也就是说 [!htp] 和 [ph!t] 没有区别。

\begin{figure}[H]
	% 居中
	\centering 
	% width=.5\textwidth 文档宽度的0.5
	% fig1图片放在img目录下,在此处引用无需img/前缀和图片格式后缀(png, jpg等)
	\includegraphics[width=.5\textwidth]{fig1} 
	% label紧接caption之后,用于引用
        \bicaption{这是一个很长很长的图}{This is a long tu}
	\label{fig:single}
\end{figure}


 
 


 