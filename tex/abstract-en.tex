%-------------------------------------------------
% FileName: abstract-en.tex
% Version: 0.1
% Date: 2023-06-25
% Description: 英文摘要
% Others: 
% History: origin
%------------------------------------------------- 

% 以下不用改动-------------------------------------
% 断页
\clearpage
% 加入书签, bm@ABSTRACTNAME要唯一
\phantomsection %创建跳转锚点 否则会跳转错误
\addcontentsline{toc}{chapter}{\defABSTRACTNAME}
% \addcontentsline{toc}{chapter}{Abstract}
% \chapter*{} 表示不编号,不生成目录
% \markboth{}{} 用于页眉
% 此处以英文题目作为章题目
{
    \ctexset{
        chapter={
            format=\bfseries\zihao{3}\centering,
            afterskip={24pt},
        },
    }
    \chapter*{\MakeUppercase\defTitleEn}
}


% 修改摘要和关键词---------------------------------
% 英文摘要
% 英文摘要与中文摘要的内容应一致。
\ABSTRACT{
    With the rapid advancement of 5G/6G technology and the Internet of Things, the electromagnetic environment has become increasingly complex, posing unprecedented challenges to electromagnetic signal identification and location information security. In complex spectral communication systems, electromagnetic signals are typically modulated onto high-frequency carriers for transmission, making modulation type recognition a critical aspect of electromagnetic signal identification. In radar applications, electromagnetic signals generated by radar systems can be utilized to detect and identify small-scale target devices, thereby ensuring the security of location information.
    
    However, most existing research primarily focuses on extracting representative signal features from a time-series perspective or through time-frequency transformation. The reliance on single-domain time-series feature representation presents inherent limitations in deep learning-based methods, particularly in terms of insufficient exploration of data relationships, which restricts further performance improvements. To address these challenges, it is essential to adopt a multi-dimensional signal representation approach, incorporating techniques such as image-based analysis and complex network modeling. By leveraging graph mapping techniques and complex network theory, researchers can transform signal data into network graph representations, enabling in-depth exploration of hidden relationships within the signal data through advanced graph analysis methods. Studying the topological structure of signals not only effectively unveils structural characteristics within signal sequences but also significantly enhances recognition performance. This paper explores graph mapping algorithms in the context of electromagnetic signal modulation recognition and target identification. The primary research contributions are as follows:
    \begin{enumerate}
         \item  For the task of target recognition in harmonic radar detection scenarios, this paper proposes a framework named AgmNet. The framework consists of modules such as feature curve generation, feature sequence mapping to complex networks, and inter-band feature mining. This paper first introduces an attention graph mapping method based on the self-attention or cross-attention mechanism, which transforms feature sequences into graph structures that are particularly well-suited for feature sequences with weak time dependence. Next, the global, inter-band, and intra-band relationships between the two L\&S bands are modeled and integrated with a simple graph convolution network to form an end-to-end classification framework, enabling the effective fusion of two-band features. This approach was evaluated using the L\&S two-band dataset collected from the harmonic radar detection platform, demonstrating its effectiveness.
        \item  For signal modulation identification tasks involving labeled data, this paper introduces a novel signal re-representation method, PSGformer, which incorporates a phase-space recursive network to establish a mapping relationship between time series and graph networks, enabling the application of graph-based data mining techniques. To reduce the threshold dependence of the recursive network adjacency matrix and enhance graph mapping efficiency, PSGformer leverages the Transformer attention mechanism to enable the parallel processing of graph construction and graph convolution. Additionally, considering the system’s sensitivity to noise and the low-rank issue of the linear attention mechanism during graph mapping, a deep contraction convolution network is introduced as a bypass compensation mechanism within the attention framework. Finally, the signal data is preprocessed using a frequency selection module. Through end-to-end training, PSGformer efficiently performs the modulation identification task and is evaluated on four modulated signal datasets, demonstrating its effectiveness.
        \item  For signal modulation recognition in scenarios without labeled data, this paper proposes a graph-based unsupervised signal representation learning framework (SGRL). This framework jointly optimizes feature representation through global and local feature comparison, as well as instance-level comparison. In SGRL, a recursive graph model is first utilized to learn global features, followed by a simplified view model to extract local features. A graph contrastive loss function is then applied to bring global and local features closer together, ultimately obtaining graph-level comparative representations. Next, a twin network based on the recursive graph model is employed to create instance-level comparison branches, while a data augmentation technique is used to generate positive and negative samples for contrastive learning, thereby enhancing the model's feature clustering capability. Finally, the invariant characteristics of the signal are extracted using a pre-trained model and utilized for downstream classification and clustering tasks. This approach was evaluated on two modulated signal datasets, demonstrating its effectiveness.
         
    \end{enumerate}
    
    In summary, this paper addresses various electromagnetic signal classification tasks by innovating graph mapping methods for electromagnetic signals and designing a series of auxiliary modules. This approach not only enhances the model's recognition performance but also maintains parameter efficiency and linear computational complexity, thereby improving the generalization capability of the graph mapping methods.
}

% 英文关键词
% 每一个英文关键词都必须与中文关键词一一对应。
\KEYWORDS{modulation recognition, target recognition, harmonic radar, graph mapping algorithm, comparison learning}




 

