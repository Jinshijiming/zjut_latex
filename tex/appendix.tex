% hyperef 精确定位
% 设置一个anchor,主要针对\addcontentsline
% 防止目录,书签等指向错误位置 
% \phantomsection
% 增加到目录,与chapter同级别
% \addcontentsline{toc}{chapter}{\appendixname} 
% \chapter*{} 表示不编号,不生成目录
% \markboth 用于页眉
% \chapter*{\appendixname \markboth{\appendixname}{}} 

% appendix 用于附录章节的特殊编号 从A开始

\newcommand{\name}{作者简介}

% 附录章
% 断页
\clearpage
% hyperef 精确定位
% 设置一个anchor,主要针对\addcontentsline
% 防止目录,书签等指向错误位置 
\phantomsection
%增加到目录,与chapter同级别
\addcontentsline{toc}{chapter}{\name}
% \chapter*{} 表示不编号,不生成目录
% \markboth{}{} 用于页眉
\chapter*{\name \markboth{\name}{}}

{
    %设置列表项 无缩进,悬挂缩进,编号,对齐到段落左端
    \setlist[enumerate,1]{label=[\arabic*],itemindent=!,align=parleft,left=0pt}
    %设置一个局部section命令 
    %\titleformat{⟨command⟩}[⟨shape⟩]{⟨format⟩}{⟨label⟩}{⟨sep⟩}{⟨before⟩}[⟨after⟩]
    %sep设置标题与编号的间距
    \titleformat{\section}{\zihao{4}\heiti}{\arabic{section}}{1ex}{}
    \setcounter{section}{0}

    \section{作者简历} 
    2015年9月——2019年6月,中北大学能源与动力工程专业,攻读工学学士学位。

    2022年9月——2025年6月,浙江工业大学信息工程学院控制工程专业,攻读工学硕士学位。

    \section{攻读硕士学位期间发表的学术论文} 
    \begin{enumerate}
        \item Y. Bu, J. Hu, C. Chen, S. Bai, et al. ResNet incorporating the fusion data of RGB \& hyperspectral images improves classification accuracy of vegetable soybean freshness, Sci Rep 14 (2024) 2568.
        \item S. Bai, Z. Chen, S. Zhou, et al. Transformer-Based Phase Space Graph Convolutional Network for Modulation Recognition[J]. IEEE Transactions on Cognitive Communications and Networking. (中科院SCI一区,外审中,一作)
        \item S. Bai, Y. Ge, K. Li, et al. Attention Graph Mapping Network for Electronic Device Classification of Harmonic Signals[J]. IEEE Transactions on Information Forensics and Security. (中科院SCI一区,外审中,一作)
        \item 徐东伟,李可兴,白松航,等. 基于自适应隐写触发器的中毒调制信号生成方法[J]. 微电子学与计算机. (外审中,学生二作)
    \end{enumerate}

    \section{参与的科研项目及获奖情况}
    \begin{enumerate}
        \item	2022年度新生奖学金
        \item	2022/2023年度校研究生二等奖学金
        \item	2023/2024年度校研究生二等奖学金
        \item	CCF BDCI 调制识别赛题一等奖
        \item	华为人工智能培训证书
        \item	谐波雷达电子设备检测技术,横向项目,编号:JG-WA-2023022
        \item	电磁信号识别智能xxxxxx技术,纵向项目,编号:JG-WA-2023030
    \end{enumerate}

    \section{发明专利}
    \begin{enumerate}
        \item	宣琦,白松航,陈壮志,徐东伟,周树锦. 一种基于Transformer和递归图的信号调制识别方法及装置[P]。中国,CN202411737058.4
        \item	尹鹏,葛一正,陈壮志,刘义伟,白松航. 基于子图融合的谐波雷达目标识别方法、设备、存储介质[P]。中国,CN202410471272.3,2024.09.24
    \end{enumerate}
}


